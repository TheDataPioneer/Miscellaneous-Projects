\documentclass{beamer}
\usepackage{graphicx}
\usepackage{paralist}
\usepackage{outlines}

\title{Glass Glare}
\author{Mendocino College - Digital Image Manipulation with Photoshop}
\titlegraphic{\vspace{-10mm}\includegraphics[width = .9\textwidth]{images/photoshop.jpg}} 
\date{\vspace{-5em}} 


\mode <presentation>
\usetheme{Warsaw}
\usecolortheme{default}

\setbeamerfont{footline}{size=\fontsize{5}{8}\selectfont}

\definecolor{darkred}{rgb}{20,0,0}
\definecolor{darkgreen}{RGB}{40,110,20}
\definecolor{darkpurple}{RGB}{30,0,30}
\definecolor{chardonnay}{RGB}{255, 255, 204}

\setbeamercolor*{palette primary}{fg=white, bg=darkgreen}


\begin{document}
	{
		\setbeamertemplate{footline}{} 
		\setbeamertemplate{headline}{} 
		\begin{frame}
			\vspace{-35pt}
			\maketitle
		\end{frame}
	}
		
		
\section{What is Glass Glare?}

\subsection{What is Glass Glare?}		

	\begin{frame}
		\frametitle{What is Glass Glare?}
		\begin{outline}
			\1 Glasses glare happens when light reflects off the lenses of someone’s glasses in a photograph. 
			\1 Whether you took your picture with a flash or it was just a sunny photoshoot, glasses glare can be an unwanted and distracting addition to your picture. 
			\1 There are several options to get rid of glasses glare without compromising your original image. 
		\end{outline}
	\end{frame}

\section{How to remove Glass Glare}
\subsection{How to remove Glass Glare}
	\begin{frame}
	\frametitle{How to remove Glass Glare}
	\begin{outline}
		\1 Use a selection tool to select the glasses with the glare you wish to correct.
		\1 Use the Clone Stamp Tool to take one part of glasses and paint it over the part of the glasses with glare.
		\1 Then you can use the Healing Brush Tools to finish touching up the glasses.
		\1 Mask the selection, then feather and blur the edges to help blend the pixels with the adjacent pixels.
		\1 Use Adjustment Layers and tweak the settings until the glare fades away.
		\1 Finish up with the Camera Raw Filter to correct bright, hazy glare.  
		\2 Adjust the contrast and brightness of your image to match the glare and make it less visible.  
	\end{outline}
\end{frame}

\section{Example}

\subsection{Example}		
	\begin{frame}
		\frametitle{Glass Glare Example}
		\begin{center}
			\includegraphics[width=1.0\textwidth]{images/remove glass glare.jpg}
		\end{center}
	\end{frame}

\section{Resources}
\subsection{Additional Resources for the Glass Glare}		
	\begin{frame}
		\frametitle{Additional Resources for the Glass Glare}
		\begin{outline}
			\1 How to remove glare from glasses in Adobe Photoshop.
			\2  By:  Adobe
			\2 https://www.adobe.com/products/photoshop/remove-glasses-glare.html
			\1 How to Remove Glare From Glasses in Photoshop
			\2  By:  Craig Boheman
			\2 https://www.makeuseof.com/photoshop-remove-glare-from-glasses-how-to/
			\1 HOW TO REMOVE REFLECTIONS FROM GLASSES IN PHOTOSHOP
			\2  By:  Colin Smith
			\2 	https://photoshopcafe.com/remove-reflections-glasses-photoshop/
		\end{outline}
	\end{frame}

	
\end{document}